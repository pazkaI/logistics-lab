\documentclass[ngerman]{tudscrreprt}
\iftutex
\usepackage{fontspec}
\else
\usepackage[T1]{fontenc}
\usepackage[ngerman=ngerman-x-latest]{hyphsubst}
\fi
\usepackage{babel}
\usepackage[german]{isodate}

\newcommand{\code}[1]{\texttt{#1}}

\begin{document}
\faculty{Fakultät Maschinenwe­sen}
\institute{Institut für Technische Logistik und Arbeitssyste­me}
\chair{Professur für Technische Logistik}
\date{2022-03-05}
\author{%
Pascal%
\matriculationnumber{12345678}%
\and%
Nico%
\matriculationnumber{87654321}%
\and%
Ferdinand Thiessen%
\matriculationnumber{87654321}%
}
\title{Logistics Lab}
\subtitle{Bericht Gruppe 6}
\maketitle

\tableofcontents

\chapter{Aufgabe 1}
Aufgabenstellung: Erstellen Sie ein Konzept zur Berechnung eines gültigen sowie möglichst optimalen Einsatzplanes der Fahrzeuge.

\chapter{Aufgabe 2}
\section{Initiale Probleme}
Bei der Umsetzung von Aufgabe 2 wurde zur Programmierung von der hauseigenen Lego Software abgesehen.

Leider wollte uns die Benutzung des \code{repeat until}-Blockes, welcher eine while-Schleife abbilden soll, nicht gelingen.
Sowohl die Bedingung $x < y$ als auch $x > y$ führten nicht zum gewünschten Verhalten.
Nur bei $x = y$ zeigte der Roboter eine Reaktion.

Da das Programm die Motoren Rotation allerdings in zu großen Abständen abtastet, wurde nie exakt die 2-Meter-Marke gemessen und somit überfahren.
Aufgrund dessen wurde nach einer Alternative gesucht. Der Hinweis der Tutoren, die alte Softwareversion zu benutzen, kam leider zu spät.

Das MicroPython Package, welches ebenfalls von Lego bereitgestellt wird, bot sich hierfür bestens an.
Außerdem ermöglichte das eine einfachere Einarbeitung durch Python-Vorkenntnisse, verständlicheren Code und Kollaboration durch Git.


\section{Experiment 1: Freie Fahrt}
\subsection{Aufgabenstellung}
Das Fahrzeug soll ohne weitere Sensorik eine Strecke von 2 m geradeaus
zurücklegen

\subsection{Herangehensweise}
Die Geschwindigkeit des Motors setzen wir initial auf $440$, da dies scheinbar die Maximalgeschwindigkeit ist.

Um die zurückgelegte Strecke des Roboters zu ermitteln, muss die Rotation des Motors gemessen werden.
Mit Hilfe der Funktion \code{motor.angle()} wird die Motorstellung, als Winkel, ausgelesen und kann
mit Hilfe des Radumfangs in eine Strecke umgerechnet werden.

\chapter{Aufgabe 3}

\end{document}
